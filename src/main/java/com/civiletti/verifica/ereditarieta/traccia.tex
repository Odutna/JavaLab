\documentclass{article}
\usepackage{amsmath}
\usepackage{graphicx, multicol}
\usepackage[a4paper, top=1cm, bottom=2cm, left=2.5cm, right=2.5cm]{geometry}

%\usepackage{titling}

\setlength{\droptitle}{-2cm}  % Riduce lo spazio sopra il titolo

\title{}

\author{Ing. Civiletti}
\date{16/02/2025}

% Aggiunta di informazioni sopra il titolo
\posttitle{%
\begin{center}
\Large \textbf{Gestione delle Figure Geometriche nel Piano Cartesiano}\\ % Testo sopra il titolo
\vspace{0.3cm} % Spazio tra la verifica e il titolo
\end{center}
}

% Aggiunta di informazioni sotto il titolo
\pretitle{%
\begin{center}

        \hspace{-0.5cm} Classe:  \textbf{4A INF} | Nome: \underline{\hspace{2cm}}  Cognome: \underline{\hspace{2cm}} 
        Data: \underline{\hspace{0.5cm}}/\underline{\hspace{0.5cm}}/\underline{\hspace{1cm}} % Data con spazi vuoti
        
    \end{center}
}

\begin{document}

\maketitle
\vspace{-2.5cm} % Riduce lo spazio sotto il titolo
\section*{Introduzione}
Sviluppa un'applicazione Java per la gestione di figure geometriche nel piano cartesiano. L'applicazione, ricevute in input le coordinate dei punti che definiscono una figura (3 punti per un triangolo, 4 punti per un quadrilatero, ecc.), dovrà calcolare perimetro e area.  A tale scopo, si utilizzi la seguente formula per il calcolo dell'area di un poligono:

\section*{Formula dell'area di un poligono}
L'area di un poligono con $n$ vertici può essere calcolata utilizzando la formula del determinante o la formula di Gauss:

\begin{equation}
A = \frac{1}{2} \left| x_1(y_2 - y_n) + x_2(y_3 - y_1) + x_3(y_4 - y_2) + \dots + x_n(y_1 - y_{n-1}) \right|
\end{equation}

dove:
\begin{itemize}
\item $(x_1, y_1), (x_2, y_2), \dots, (x_n, y_n)$ sono le coordinate dei vertici del poligono, ordinate in senso orario o antiorario.
\item Il valore assoluto assicura che l'area sia sempre positiva.
\end{itemize}

\noindent In Java, questa formula può essere implementata come segue:

\begin{verbatim}
double area = 0;
for (int i = 0; i < vertici.length; i++) {
int j = (i + 1) % vertici.length;
area += vertici[i].getX() * vertici[j].getY();
area -= vertici[i].getY() * vertici[j].getX();
}
return Math.abs(area) / 2;
\end{verbatim}

\section*{Esempio di utilizzo delle classi}

\begin{verbatim}
// Test delle figure
Punto p1 = new Punto(0, 0);
Punto p2 = new Punto(3, 0);
Punto p3 = new Punto(3, 4);
Punto p4 = new Punto(0, 4);

    Triangolo triangolo = new Triangolo(p1, p2, p3);
    System.out.println("Area triangolo: " + triangolo.calcolaArea());
    System.out.println("Perimetro triangolo: " + triangolo.calcolaPerimetro());

    // Test rettangolo
    Rettangolo rettangolo = new Rettangolo(p1, p2, p3, p4);
    System.out.println("Area rettangolo: " + rettangolo.calcolaArea());
    System.out.println("Perimetro rettangolo: " + rettangolo.calcolaPerimetro());

    // Test quadrato
    Punto q1 = new Punto(0, 0);
    Punto q2 = new Punto(2, 0);
    Punto q3 = new Punto(2, 2);
    Punto q4 = new Punto(0, 2);
    Quadrato quadrato = new Quadrato(q1, q2, q3, q4);
    System.out.println("Area quadrato: " + quadrato.calcolaArea());
    System.out.println("Perimetro quadrato: " + quadrato.calcolaPerimetro());
\end{verbatim}


\section{Classe Punto}
La classe \texttt{Punto} rappresenta un punto nel piano cartesiano $(x, y)$ con le seguenti caratteristiche:
\begin{itemize}
\item Due variabili private di tipo \texttt{double}: $x$ e $y$.
\item Un costruttore che accetti le coordinate $x$ e $y$.
\item Un costruttore di copia che crei un nuovo punto a partire da un punto esistente.
\item Metodi getter e setter per le coordinate.
\item Un metodo \texttt{distanza(Punto p)} che calcoli la distanza tra due punti.
\item Un metodo \texttt{equals(Punto p)} per verificare se due punti coincidono.
\item Un metodo \texttt{toString()} che restituisca una rappresentazione testuale del punto.
\end{itemize}








\section{Classe FiguraGeometrica}
La classe \texttt{FiguraGeometrica} deve:
\begin{itemize}
\item Contenere un array di oggetti \texttt{Punto} per memorizzare i vertici della figura.
\item Includere un costruttore che accetti un array di punti.
\item Fornire un metodo per calcolare il perimetro della figura.
\item Fornire un metodo per calcolare l'area della figura.
\end{itemize}

\section{Classi Triangolo, Quadrato e Rettangolo}
\begin{multicols}{2}
\subsection{Classe \texttt{Triangolo}}
La classe \texttt{Triangolo} deve soddisfare i seguenti requisiti:
\begin{itemize}
\item Utilizzare la classe \texttt{Punto} per rappresentare i tre vertici.
\item Includere un costruttore che accetti tre oggetti \texttt{Punto}.
\item Utilizzare il costruttore di copia della classe \texttt{Punto} per memorizzare i vertici.
\item Implementare un metodo per calcolare l'area del triangolo utilizzando la formula di Erone:
\end{itemize}

\[
A = \sqrt{s(s-a)(s-b)(s-c)}
\]

dove $a$, $b$, $c$ sono i lati del triangolo e $s$ è il semiperimetro:

\[
s = \frac{a + b + c}{2}
\]

\begin{itemize}
\item Implementare un metodo per calcolare il perimetro:
\end{itemize}

\[
P = a + b + c
\]

\subsection{Classe \texttt{Rettangolo}}
La classe \texttt{Rettangolo} deve:
\begin{itemize}
\item Essere una sottoclasse di \texttt{FiguraGeometrica}.
\item Avere un costruttore che accetti quattro oggetti \texttt{Punto}.
\item Implementare un metodo per calcolare l'area utilizzando la formula:

\[
A = \text{base} \times \text{altezza}
\]

dove la base e l'altezza possono essere determinate calcolando la distanza tra i punti adiacenti.
\item Implementare un metodo per calcolare il perimetro:

\[
P = 2 \times (\text{base} + \text{altezza})
\]

\end{itemize}

\subsection{Classe \texttt{Quadrato}}
La classe \texttt{Quadrato} deve:
\begin{itemize}
\item Essere una sottoclasse della classe \texttt{Rettangolo}.
\item Includere un costruttore che accetti quattro oggetti \texttt{Punto}.
\item Implementare un metodo per calcolare l'area usando la formula:

\[
A = \text{lato}^2
\]

    \item Implementare un metodo per calcolare il perimetro:

\[
P = 4 \times \text{lato}
\]
\end{itemize}

\end{multicols}









\begin{multicols}{2}


\section{Esempi di Test}

\subsection{Test Triangolo Rettangolo}
Punti di input:
\begin{itemize}
\item \( P_1(0, 0) \)
\item \( P_2(3, 0) \)
\item \( P_3(3, 4) \)
\end{itemize}

Output atteso:
\begin{verbatim}
Area triangolo: 6.0
Perimetro triangolo: 12.0
\end{verbatim}

\subsection{Test Rettangolo}
Punti di input:
\begin{itemize}
\item \( P_1(0, 0) \)
\item \( P_2(3, 0) \)
\item \( P_3(3, 4) \)
\item \( P_4(0, 4) \)
\end{itemize}

Output atteso:
\begin{verbatim}
Area rettangolo: 12.0
Perimetro rettangolo: 14.0
\end{verbatim}

\vspace{1cm} % Spazio prima dell'elenco
\subsection{Test Quadrato}
Punti di input:
\begin{itemize}
\item \( Q_1(0, 0) \)

    % \vspace{2cm}
    \item \( Q_2(2, 0) \)
    
    \item \( Q_3(2, 2) \)
    \item \( Q_4(0, 2) \)
\end{itemize}

Output atteso:
\begin{verbatim}
Area quadrato: 4.0
Perimetro quadrato: 8.0
\end{verbatim}

\subsection{Test Pentagono Irregolare}
Punti di input:
\begin{itemize}
\item \( V_1(0, 0) \)
\item \( V_2(2, 1) \)
\item \( V_3(3, 3) \)
\item \( V_4(1, 4) \)
\item \( V_5(-1, 2) \)
\end{itemize}

Output atteso:
\begin{verbatim}
Area pentagono: 9.0
\end{verbatim}

\noindent\textbf{Dimostrazione matematica:}
\[
A = \frac{1}{2} \left|
0(1-2) + 2(3-0) + 3(4-1) + 1(2-3) + (-1)(0-4)
\right|
\]
\[
A = \frac{1}{2} \left|
0(-1) + 2(3) + 3(3) + 1(-1) + (-1)(-4)
\right|
\]
\[
\hspace{-2cm} A = \frac{1}{2} \left| 0 + 6 + 9 - 1 + 4 \right| = \frac{18}{2} = 9.0
\]

\end{multicols}



\end{document}
